\documentclass[12pt, letterpaper, twoside]{article}
\usepackage[legalpaper, portrait, margin=0.8in]{geometry}
\usepackage{soul}
\usepackage[norule]{footmisc}
\usepackage{multicol,caption}
\usepackage{sectsty}
\usepackage{graphicx}
\usepackage{subcaption}
\usepackage{titling}
\usepackage[toc,page]{appendix}
\usepackage{hyperref}
\usepackage[T1]{fontenc}
\renewcommand{\thesection}{\Roman{section}} 
\renewcommand{\thesubsection}{\thesection.\Roman{subsection}}
\graphicspath{ {./images/} }
\usepackage{amsfonts}
\usepackage{array}
\usepackage{tabu}
\usepackage[table]{xcolor}
\usepackage{pdflscape}
\usepackage{makecell}
\renewcommand\theadalign{bc}
\renewcommand\theadfont{\bfseries}
\renewcommand\theadgape{\Gape[4pt]}
\renewcommand\cellgape{\Gape[4pt]}
\usepackage{longtable}
\usepackage{hyperref}
\usepackage{amssymb}
\usepackage{bm}
\usepackage{pgfplots}
\usepackage{pgfplotstable}
\pgfplotsset{compat=1.7}
\usepackage{tikz,lipsum,lmodern}
\usepackage[most]{tcolorbox}
\usepackage{subfiles}
\usepackage{afterpage}
\usepackage{filecontents}
\usepackage{setspace}
\usepackage{caption}
\usepackage{minted}
\usepackage{fancyhdr}
\DeclareCaptionType{equ}[][]
\newenvironment{Figure}
  {\par\medskip\noindent\minipage{\linewidth}}
  {\endminipage\par\medskip}
\usepackage{titling}
\setstretch{1.25}
\usepackage{enumitem,amssymb}
\newlist{todolist}{itemize}{2}
\setlist[todolist]{label=$\square$}

%\pagenumbering{gobble}
\setlength{\droptitle}{-5em}
\setlength{\parindent}{1em}
\setlength{\parskip}{0.9em}

\sectionfont{\fontsize{11}{15}\selectfont}
\subsectionfont{\fontsize{11}{15}\selectfont}
\setlength{\columnsep}{1cm}

\pagestyle{fancy}
\rhead{PHY294}

\author{\textbf{James Li}}
\title{\textbf{PHY294: Quantum and Thermal Review Notes}\vspace{-1.5em}}
\author{James Li\vspace{-2em}}
\date{\today}

\begin{document}

\maketitle
\vspace{-2em}
\section{The Three Dimensional Schr\"{o}dinger's Equation}

\begin{itemize}
    \item From PHY293, we know that the time-independent, one-dimesional Schr\"{o}diger's Equation is:
    \begin{equation}
        \frac{d^2\psi}{dx^2} = \frac{2M}{\hbar^2}[U-E]\psi
    \end{equation}
    where U is the potential energy of the particle and M is the mass.
    \item In 3 dimensions, since we know that the wavefunction and the potential energy of the particle depend on the position of the particle:
    \begin{equation*}
        \psi(x,y,z) = \psi(\vec{r}), U(x,y,z) = U(\vec{r})
    \end{equation*}
    \item As such, the 3-D SE is:
    \begin{equation}
        \frac{\partial^2\psi}{\partial x^2} + \frac{\partial^2\psi}{\partial y^2} + \frac{\partial^2\psi}{\partial z^2} = \frac{2M}{\hbar^2}[U-E]\psi(\vec{r})
    \end{equation}
\end{itemize}

\subsection{Two-Dimensional Square Well}

\begin{itemize}
    \item Consider a rigid, square box where the potential of the particle is 0 inside this box (U = 0):

    \begin{align}
        \vcenter{\hbox{\includegraphics[width=4cm,height=4cm]{2D well.png}}}
        &\qquad\qquad
        \begin{aligned}
            U(\vec{r}) = \begin{cases}
            0, \hspace{0.5cm} 0\leq x\leq a \\
            \infty, \hspace{0.5cm}  \text{else}
            \end{cases}
        \end{aligned}\\
        \vcenter{\hbox{\begin{minipage}{4cm}
        \end{minipage}}}
        & \notag
    \end{align}
    \item As such, energy E will be all kinetic and ranges from 0 $\leq$ E < $\infty$.
    \item Now, we can write the 2-D SE as:
    \begin{equation*}
        \frac{\partial^2\psi}{\partial x^2} + \frac{\partial^2\psi}{\partial y^2} = -\frac{2ME}{\hbar^2}\psi(\vec{r})
    \end{equation*}
    \item This equation can be solved by using separation of variables, taking the wavefunction to be a product of functions of x and y:
    \begin{equation*}
        \psi(x, y) = X(x)Y(y)
    \end{equation*}
    \item Now, the equation can be rewritten as and solved by second-order ordinary differential equation strategies:
    \begin{align}
        X"(x)Y(y) + X(x)Y"(y) &= -\frac{2ME}{\hbar^2}X(x)Y(y) \\
        \frac{X"(x)}{X(x)} + \frac{Y"(y)}{Y(y)} &= -\frac{2ME}{\hbar^2} = -k^2 \text{(constant)} 
    \end{align}
    \item Using the boundary conditions found from the two-dimensional well, namely $X(0) = X(a) = Y(0) = Y(a) = 0$, we can solve the wavefunction to be:
    \begin{equation}
        \psi(x,y) = BC\sin{\frac{n_x\pi x}{a}}\sin{\frac{n_y\pi y}{a}} \footnote{Recall that in a 1D well, we have the property: $\psi_n = \sqrt{\frac{2}{L}}\sin{(\frac{n\pi x}{L})}$. Using this, we can solve for constants B, C.}
    \end{equation}
    where $n_x, n_y$ are positive integers.
\end{itemize}

\subsubsection{Allowed Energies}

\begin{align*}
    \frac{X"(x)}{X(x)} + \frac{Y"(y)}{Y(y)} &= -\frac{2ME}{\hbar^2} \\
    \frac{n_x^2\pi^2}{a^2} + \frac{n_y^2\pi^2}{a^2} &= -\frac{2ME}{\hbar^2} \\
    E_{n_x,n_y} &= \frac{\hbar^2\pi^2}{2Ma^2}(n_x^2 + n_y^2)
\end{align*}

\begin{itemize}
    \item This is the allowed energies given the two positive integers, $n_x, n_y$.
\end{itemize}
\subsubsection{Degeneracy}

\begin{itemize}
    \item In a two-dimensional box, there can be different wavefunctions where the particle has the same energy, for example $\psi_{12}, \psi_{21} \longrightarrow E_{12} = E_{21} = 5E_0$.
    \item In general, if there are N independent wave functions with same energy E, we say that the energy level E is \textbf{degenerate} with \textbf{degeneracy} N.
\end{itemize}

\subsection{Central Forces (2D Polar Schr\"{o}dinger's Equation)}

\begin{itemize}
    \item Many physical systems are defined such as the potential energy of the particle is dependent on its distance to the center, r, rather than its direction. As such, we can rewrite it as:
    \begin{equation*}
        \psi = \psi(r, \phi)
    \end{equation*}
    \begin{equation}
        \frac{\partial^2\psi}{\partial r^2} + \frac{1}{r}\frac{\partial\psi}{\partial r} + \frac{\partial^2\psi}{\partial \phi^2} = \frac{2M}{\hbar^2}[U(r) - E]\psi(\vec{r})
    \end{equation}
    \item Using the same strategy of separation of variables, we obtain that:
    \begin{align*}
        \frac{\Phi"(\phi)}{\Phi(\phi)} &= - \frac{r^2R"(r) + rR'(r)}{R(r)} + \frac{2Mr^2}{\hbar^2}[U(r) - E]\\
        \Phi"(\phi) &= -m^2\Phi(\phi) \\
        R" + \frac{R'}{r} - \Big[\frac{m^2}{r^2} + \frac{2M}{\hbar^2}(U - E)\Big]R &= 0
    \end{align*}
    \item The general solution for $\Phi(\phi)$ is a combination of $\sin{m\phi}$ and $\cos{m\phi}$, where m must be an integer:
    \begin{equation*}
        \Phi(\phi) = \cos{(m\phi)} + i\sin{(m\phi)} = e^{im\phi}
    \end{equation*}
\end{itemize}
\subsubsection{Quantizing Angular Momentum}

\begin{itemize}
    \item So far, we can conclude that:
    \begin{equation*}
        \psi(r, \phi) = R(r)e^{im\phi}
    \end{equation*}
    \begin{align}
        \vcenter{\hbox{\includegraphics[width=4cm,height=4cm]{circle.png}}}
        &\qquad\qquad
        \begin{aligned}
            \psi(r, \phi) \propto e^{im\phi} = e^{i(m/r)s}
        \end{aligned}\\
        \vcenter{\hbox{\begin{minipage}{4cm}
        \end{minipage}}}
        & \notag
    \end{align}
    \item Given a particle has momentum $\hbar k$, the angular momentum, L, of the particle is:
    \begin{equation*}
        L = p_{tang}r = m\hbar
    \end{equation*}
    which shows that angular momentum is now quantized as multiples of $\hbar$.
\end{itemize}

\subsubsection{Energy Levels}

\begin{itemize}
    \item Using the radial equation, we can note that for a different value of m, we get different allowed energies. In addition, since the m dependence is only found as $\frac{m^2}{r^2}$, meaning that we will have the same energies whether $\pm m$:
    \begin{equation*}
        E_{n,m} = E_{n, -m}
    \end{equation*}
    \item This also means that except when m = 0, the energy level $E_{n,m}$ is \textbf{twofold degenerate}.
\end{itemize}

\subsection{Three-Dimensional Central-Force Problem}

\begin{itemize}
    \item Similar to two dimensions where we use polar coordinates, we now need an r to represent the distance to the central force, so we use \textbf{spherical coordinates}.
    \begin{align}
        \vcenter{\hbox{\includegraphics[width=5cm,height=4cm]{sphere.png}}}
        &\qquad\qquad
        \begin{aligned}
            x = r\sin{\theta}\cos{\phi} \\
            y = r\sin{\theta}\sin{\phi} \\
            z = rcos{\theta}
        \end{aligned}\\
        \vcenter{\hbox{\begin{minipage}{4cm}
        \end{minipage}}}
        & \notag
    \end{align}
    \item The three-dimensional Schr\"{o}dinger's Equation in spherical coordinates can be written as:
    \begin{equation}
        \frac{1}{r}\frac{\partial^2}{\partial r^2}(r\psi) + \frac{1}{r^2\sin{\theta}}\frac{\partial}{\partial\theta}(\sin{\theta}\frac{\partial\psi}{\partial\theta}) + \frac{1}{r^2\sin^2{\theta}\frac{\partial^2\psi}{\partial\phi^2}} = \frac{2M}{\hbar^2}[U(r) - E]\psi
    \end{equation}
    \item Once again, using the separation of variables technique ($\psi = R(r)\Phi(\phi)\Theta(\theta)$), we obtain:
    \begin{align*}
        \frac{\Phi"(\phi)}{\Phi(\phi)} &= f(r, \theta) \\
        \Phi"(\phi) &= -m^2\Phi(\phi) \\
        \Phi(\phi) &= e^{im\phi}
    \end{align*}
    and
    \begin{align*}
        \frac{1}{\sin\theta}\frac{d}{d\theta}(\sin\theta\frac{d\Theta}{d\theta}) + (k - \frac{m^2}{\sin^2\theta})\Theta &= 0 \\
        \frac{2M}{\hbar^2}[U(r) + \frac{k\hbar^2}{2Mr^2} - E](rR) &= \frac{d^2}{dr^2}(rR)
    \end{align*}
\end{itemize}

\subsubsection{Quantizing Angular Momentum (again)}

\begin{itemize}
    \item Using the same idea as the 2D quantization of angular momentum:
    \begin{align}
        \vcenter{\hbox{\includegraphics[width=5cm,height=4cm]{cylinder.png}}}
        &\qquad\qquad
        \begin{aligned}
            \phi = \frac{s}{\rho}
        \end{aligned}\\
        \vcenter{\hbox{\begin{minipage}{4cm}
        \end{minipage}}}
        & \notag
    \end{align}
    \begin{equation*}
        \Phi(\phi) = e^{im\phi} = e^{i(m/\rho)s}
    \end{equation*}
    \item Knowing that the tangential momentum $p_{tang} = \frac{\hbar m}{\rho}$, we can calculate $p_{tang}$ by the radius:
    \begin{equation*}
        L_z = m\hbar
    \end{equation*}
    \item Now, dealing with $\Theta(\theta)$, we observe that the only acceptable solution for k is:
    \begin{equation*}
        k = l(l+1)
    \end{equation*}
    where l is a positive integer greater than or equal in magnitude to m ($l \geq |m|$).
    \item This way, we can define the magnitude of angular momentum and represent it using the vector model as:
    \begin{align}
        \vcenter{\hbox{\includegraphics[width=5cm,height=4cm]{vector model.png}}}
        &\qquad\qquad
        \begin{aligned}
            L = \sqrt{l(l+1)}\hbar
        \end{aligned}\\
        \vcenter{\hbox{\begin{minipage}{4cm}
        \end{minipage}}}
        & \notag
    \end{align}
    \item The Heisenberg uncertainty principle extends to angular momentum and implies that no two components of \textbf{L} can simultaneously have definite values.
\end{itemize}
\subsection{Energy Levels of The Hydrogen Atom}

\begin{itemize}
    \item Finally, we have to consider the radial equation, $R(r)$ since it is the only equation with E in it.
    \begin{equation*}
        \frac{d^2}{dr^2}(rR) = \frac{2M}{\hbar^2}[U(r) + \frac{l(l+1)\hbar^2}{2Mr^2} - E](rR)
    \end{equation*}
    \item The allowed values of E will depend on the angular momentum quantum number, l, and we can predict that in a central-force problem, a level with $L = \sqrt{l(l+1)}\hbar$ will be at least $(2l+1)$-fold degenerate.
    \item For the Hydrogen atom, we take an electron bound to a proton, where the potential energy function is:
    \begin{equation*}
        U(r) = \frac{-e^2}{4\pi\varepsilon_0r}
    \end{equation*}
    \item After plugging $U(r)$ into the radial equation and solving for E, we obtain:
    \begin{equation*}
        E = -\frac{m_e(ke^2)^2}{2\hbar^2}\frac{1}{n^2}\footnote{This first term is the Rydberg energy, equivalent to 13.6 eV.} = - \frac{E_R}{n^2}
    \end{equation*}
    \begin{align}
        \vcenter{\hbox{\includegraphics[width=7cm,height=5cm]{energy level diagram.png}}}
        &\qquad\qquad
        \begin{aligned}
            L = \sqrt{l(l+1)}\hbar \\
            E = -\frac{E_R}{n^2}
        \end{aligned}\\
        \vcenter{\hbox{\begin{minipage}{7cm}
        \captionof{figure}{Energy-level diagram of the hydrogen atom. }
        \end{minipage}}}
        & \notag
    \end{align}
    \item The nth energy level has degeneracy $n^2$ because it has $l = 0, 1, ..., (n-1)$.
\end{itemize}

\subsection{Wavefunctions of Hydrogen}
\subsubsection{1s (Ground) State}

\begin{itemize}
    \item Recall, the ground state has quantum numbers, n = 1, l = 0, m = 0. It is also spherically symmetric, which means that the wavefunction is only dependent on the radius.
    \begin{equation*}
        \psi_{1s} = R_{1s}(r)
    \end{equation*}
    \item We can rewrite the radial equation as:
    \begin{equation*}
        \frac{d^2}{dr^2}(rR) = \frac{2M}{\hbar^2}[-\frac{ke^2}{r} + \frac{E_R}{n^2}](rR)
    \end{equation*}
    which can be further simplified as \footnote{Recall that Bohr radius, $a_B = \frac{\hbar^2}{m_eke^2}$, and Rydberg energy, $E_R = \frac{ke^2}{2a_B}$}:
    \begin{equation*}
        \frac{d^2}{dr^2}(rR) = \Big(\frac{1}{n^2a_B^2} - \frac{2}{a_Br}\Big)(rR)
    \end{equation*}
    \begin{align}
        \vcenter{\hbox{\includegraphics[width=7cm,height=4cm]{1s state of hydrogen.png}}}
        &\qquad\qquad
        \begin{aligned}
            R_{1s}(r) = Ae^{-r/a_B}
        \end{aligned}\\
        \vcenter{\hbox{\begin{minipage}{7cm}
        \captionof{figure}{Wavefunction of 1s state of hydrogen.}
        \end{minipage}}}
        & \notag
    \end{align}
    \item The probability of finding an electron in a small spherical volume, namely between r and r + dr, is:
    \begin{equation*}
        \mathbb{P}(r \leq \rho \leq r + dr) = |\psi|^2 dV = |R(r)|^24\pi r^2 dr
    \end{equation*}
    \item The radial probability density (radial distribution) is:
    \begin{align*}
        P(r) &= 4\pi r^2|R(r)|^2 \\
        P_{1s}(r) &= 4\pi A^2r^2e^{-2r/a_B}
    \end{align*}
    \item To solve for constant A, we can use the property of a probability density function \footnote{For density function $f(x)$, $\int_{-\infty}^{\infty}f(x) = 1$.} to obtain $A = \frac{1}{\sqrt{\pi a_B^3}}$.
\end{itemize}

\subsubsection{The 2$\mathbf{p_x}$, 2$\mathbf{p_y}$, 2$\mathbf{p_z}$ Wavefunctions}

\begin{itemize}
    \item In the l = 1 state, there are 3 2p wave functions, which corresponds to the three orientations (x, y, z). Now we cannot omit the $\Theta(\theta)$ component since $l \neq 0$, and as such, $\Theta(\theta) = \cos{\theta}$:
    \begin{equation*}
        \psi = R_{2p}(r)\cos{\theta} \longrightarrow R_{2p}(r) = Are^{-r/2a_B}
    \end{equation*}
    \item Now for the x, y and z orientations respectively, the wavefunctions are $Axe^{-r/2a_B}, Aye^{-r/2a_B}$, $Aze^{-r/2a_B}$.
\end{itemize}
\begin{figure}[!ht]
    \centering
    \includegraphics[scale = 0.75]{radial dist.png}
\end{figure}
\subsection{Shells}

\begin{figure}[!ht]
    \centering
    \includegraphics[width= \linewidth]{shell.png}
    \caption{Just remember how to draw these by heart.}
    \label{fig:my_label}
\end{figure}

\subsection{Hydrogen-Like Atoms}

\begin{itemize}
    \item Similar to how the potential energy, $U(r)$, of a hydrogen atom is $\frac{-ke^2}{r}$, the potential energy of a hydrogen-like atom is $U(r) = -\frac{Zke^2}{r}$, where Z is the number of protons in the nucleus. As such, the allowed energy of the particle is:
    \begin{equation*}
        E = -Z^2\frac{E_R}{n^2}
    \end{equation*}
\end{itemize}

\section{Electron Spin}

\subsection{Spin Angular Momentum (S)}

\begin{itemize}
    \item The total angular momentum, \textbf{J}, is the sum of the orbital angular momentum, \textbf{L}, and the spin angular momentum, \textbf{S}.
    \item Similar to the way \textbf{L} is quantized, the magnitude of \textbf{S} can also be quantized:
    \begin{equation*}
        S = \sqrt{s(s+1)}\hbar
    \end{equation*}
    where s is fixed at a value of $s = \frac{1}{2}$, meaning S is fixed at a magnitude of $S = \frac{\sqrt{3}}{2}\hbar$.
    \item Similar to orbital angular momentum, the magnitude of spin angular momentum can be represented as:
    \begin{equation*}
        S_z = m_s\hbar
    \end{equation*}
    where $m_s \in {-s, s}$. This leaves us with $m_s = \pm \frac{1}{2}$.
    \item Contrary to what was said in section I.IV, the degeneracy of the nth energy level is actually $2n^2$ because the energy of the hydrogen atom is \textbf{independent of the spin orientation}.
\end{itemize}

\subsection{Magnetic Moments}

\begin{itemize}
    \item An orbiting charge acts like a small current loop, $i$, producing a magnetic field, \textbf{B} and a torque $\mathbf{\Gamma}$.
    \begin{align}
    \vcenter{\hbox{\includegraphics[width=6cm,height=4cm]{current.png}}}
        &\qquad\qquad
        \begin{aligned}
            \mathbf{\Gamma} = i\mathbf{A} \times \mathbf{B} \\
            \mathbf{\Gamma} = \mathbf{\mu} \times \mathbf{B}
        \end{aligned}\\
        \vcenter{\hbox{\begin{minipage}{6cm}
        \end{minipage}}}
        & \notag
    \end{align}
    where $\mathbf{\mu}$ is the magnetic moment of the loop.
    \item The torque done on the current loop will produce a work which varies with the orientation of the loop:
    \begin{equation*}
        W = - \int \Gamma d\theta = -\mu B\int \sin{\theta} d\theta = \mu B\cos{\theta} + C
    \end{equation*}
    \item Potential energy U is the negative of work:
    \begin{equation}
        U = -\mu B\cos{\theta} = -\mathbf{\mu \cdot B}
    \end{equation}
    where it is at its minimum when $\theta = 0$ (or $\mu$ is parallel to $\mathbf{B}$). 
    \item The magnitude of the current loop can be defined by the charge of the electron e and its speed v ($v = \frac{2\pi r}{T}$):
    \begin{equation*}
        i = \frac{e}{T} = e\frac{v}{2\pi r}
    \end{equation*}
    \item From this we can find the gyromagnetic ratio ($\frac{\mu}{L}$):
    \begin{equation*}
        \frac{\mu}{L} = \frac{evr}{2m_evr} = \frac{e}{2m_e}
    \end{equation*}
    \item The current is travelling in a direction opposite of the electron's velocity so $\mu$ and $\mathbf{L}$ are actually antiparallel (opposite-facing).
    \begin{equation*}
        \mu = -\frac{e}{2m_e}\mathbf{L}
    \end{equation*}
\end{itemize}

\subsection{The Zeeman Effect}

\begin{itemize}
    \item \textbf{Zeeman effect:} putting an atom in a magnetic field to change its spectrum.
    \item Consider an atom where the magnetic moments due to the electrons' spins cancel out. In a state where there is zero total spin, it is called a \textbf{singlet state}.
    \item Let helium have an energy $E_0$, where its angular momentum $\textbf{L}$ has 2l+1 possible different orientations. Without a magnetic field, the energy is the same for all states, meaning $E_0$ is $(2l+1)$-fold degenerate.
    \item Now when we apply magnetic field \textbf{B}, the energy will change by $-\mu \cdot \mathbf{B}$, depending on the orientation of $\mu$. We can conclude that we remove the degeneracy of the energy level but add $2l+1$ orientations of the magnetic moment. We can denote this change in energy to be:
    \begin{align}
    \vcenter{\hbox{\includegraphics[width=9cm,height=3cm]{B field.png}}}
        &\qquad\qquad
        \begin{aligned}
            \Delta E = -\mu\cdot \textbf{B} = (\frac{e}{2m_e})\textbf{L}\cdot\textbf{B}\\
            \Delta E = m\mu_B B
        \end{aligned}\\
        \vcenter{\hbox{\begin{minipage}{8cm}
        \end{minipage}}}
        & \notag
    \end{align}
\end{itemize}

\subsection{Spin Magnetic Moments}

\begin{equation*}
    \vec{\mu}_{orb} = -\frac{e}{2m_e}\textbf{L}
\end{equation*}
\begin{itemize}
    \item Similarly, the spin magnetic moment is proportional to the spin angular momentum \textbf{S}:
    \begin{equation*}
        \vec{\mu}_{spin} \propto \textbf{S} = -\gamma\textbf{S}
    \end{equation*}
    where $\gamma$ is the spin gyromagnetic ratio.
    \item The total magnetic moment of any electron is just the sum of its orbital and spin moments:
    \begin{equation*}
        \vec{\mu}_{tot} = \vec{\mu}_{orb} + \vec{\mu}_{spin} = -\frac{e}{2m_e}(\textbf{L} + 2\textbf{S})
    \end{equation*}
\end{itemize}

\subsection{The Anomalous Zeeman Effect}

\begin{itemize}
    \item The difference between the anomalous and the normal Zeeman effect is that the spin will contribute to the splitting of the magnetic moment.
    \item We can now denote the change in energy as:
    \begin{equation*}
        \Delta E = -\vec{\mu}\cdot \textbf{B} = \frac{e}{m_e}S_zB = \pm\mu_BB
    \end{equation*}
    meaning the separation of levels is $2\mu_BB$.
    \begin{align}
    \vcenter{\hbox{\includegraphics[width=8cm,height=3cm]{b field again.png}}}
        &\qquad\qquad
        \vcenter{\hbox{\begin{minipage}{8cm}
        \end{minipage}}}
        & \notag
    \end{align}
\end{itemize}
\section{Multielectron Atoms}

\subsection{The Independent-Particle Approximation}

\begin{itemize}
    \item The force on the electron in a multielectron atom is the sum of the force of nucleus and the repulsion by the other electrons.
    \item We want to treat each electron individually, so we call its potential energy U($\vec{r}$). This is the Independent-Particle Approximation, or the IPA potential energy function.
    \item We can use this to solve wavefunctions of multielectron atoms (find a set of wavefunctions).
    \item \textbf{Gauss's Law:} if the electron is outside spherical charge Q, the electron experiences the same electric force, $F_e$, if the entire charge Q were a point charge at r = 0:
    \begin{equation*}
        F = k\frac{Qe}{r^2}
    \end{equation*}
    \item If the electron is inside the spherical charge, it experiences \textbf{no force} from the shell.
    \item If the electron is close enough to the nucleus, it will experience the force of the nuclear charge, $Ze$, where Z is the number of protons. As the electron moves away from the nucleus ($r \rightarrow \infty$), it will move past the $Z - 1$ electrons, reducing the electric force of the nucleus by $(Z - 1)e$:
    \begin{align}
    \vcenter{\hbox{\includegraphics[width=7cm,height=4cm]{force.png}}}
        &\qquad\qquad
        \begin{aligned}
             F_{out} = \frac{[Z - (Z-1)]ke^2}{r^2} = \frac{ke^2}{r^2} \\
             U(\vec{r}) = \int F_{out}dr = -\frac{ke^2}{r} (r \rightarrow \infty) \\
             U(\vec{r}) = \int F_{in}dr = -\frac{Zke^2}{r} (r \rightarrow 0)
        \end{aligned}\\
        \vcenter{\hbox{\begin{minipage}{7cm}
        \captionof{figure}{As $r \rightarrow 0$, the electron will experience nearly the entire force of the nucleus.}
        \end{minipage}}}
        & \notag
    \end{align}
    \item The general case can be expressed as $Z \approx Z_{eff}$, where it is the effective charge felt by the electron with the same properties as the force experienced by the electron.
    \begin{equation*}
        Z_{eff} \approx Z \hspace{0.35cm}(r\rightarrow0), Z_{eff} \approx 1 \hspace{0.35cm}(r\rightarrow\infty)
    \end{equation*}
\end{itemize}
\subsection{The IPA Energy Levels}

\begin{itemize}
    \item Potential U($\vec{r}$) is sufficiently like the potential energy of the hydrogen atom. The two angular equations are the same as the hydrogen atom, meaning they have the same m and l quantum numbers:  $(2l+1)$ orientations of $m = {-l, ..., l}$.
    \item As seen previously, $U(\vec{r}$) is not affected by spin, meaning each energy level has a degeneracy of at least $2(2l+1)$.
    \begin{figure}[!ht]
        \centering
        \includegraphics[scale=1.1]{multi.png}
        \caption{Radial probability distributions for hydrogen.}
        \label{fig:radial}
    \end{figure}
    \item For the 1s state (ground state), the wavefunction is very closely concentrated to the nucleus (probability density is very favored to $r = 0$), so we can use $Z_{eff} \approx Z$.
    \begin{equation*}
        E_{1s} \approx -Z^2E_R
    \end{equation*}
    \item Now for the 2s and 2p states, in a hydrogen item, their states are degenerate, whereas in a multielectron atoms, the 2s states \textbf{are lower in energy}. This is because the 2s and 2p wavefunctions are concentrated in a region where the nuclear charge is screened by electrons in the 1s state. Thus, $Z_{eff} << Z$.
    \begin{itemize}
        \item Note that in Figure \ref{fig:radial} in the 1s distribution, the electron is most likely found at $\frac{r}{a_B}$ (highest density).
        \item 2s and 2p states peak at $\frac{4r}{a_B}$ and $\frac{5.2r}{a_B}$, so they are being screened by the 1s electrons at $r = \frac{r}{a_B}$.
    \end{itemize}
    \item However, the 2s state has a secondary peak at $r << \frac{5.2r}{a_B} \approx 0$, meaning a small density of the 2s electrons have $Z_{eff} \approx Z$. Since on average the 2s electron is more greatly affected by the nuclear force than the 2p electron, it will have lower energy \textbf{\textit{(lower energy to break attractive nuclear force)}}.
    \item In general, for principal quantum number n, \textbf{states with smaller $l$ penetrate closer to the nucleus on average and are lower in energy.}
    \begin{figure}[!ht]
        \centering
        \includegraphics{h vs multi.png}
        \caption{Hydrogen energy levels versus multielectron energy levels.}
        \label{fig:h vs multi}
    \end{figure}
    \item The most probable radius (of finding an electron) can be approximated as:
    \begin{equation*}
        r_{mp} \approx \frac{n^2a_B}{Z_{\text{eff}}}
    \end{equation*}
\end{itemize}

\subsection{The Pauli Exclusion Principle}

\begin{itemize}
    \item The ground state of the atom is not necessarily found by placing all Z electrons in the 1s state.
    \item \textbf{Pauli Exclusion Principle:} no two electrons in a quantum system can occupy the same quantum state. The easiest example of this is electron configuration of helium.
    \begin{figure}[!ht]
        \centering
        \includegraphics{helium.png}
        \caption{The two electrons cannot both have quantum numbers $m_s = \frac{1}{2}$. In the right figure, the magnetic moment, $\mu$, is always 0.}
    \end{figure}
    \item Instead, the electrons can be arranged in a fashion where they are placed in the 2s shell.
    \begin{figure}[!ht]
        \centering
        \includegraphics{helium2.png}
        \caption{No spin orientation restriction for this case.}
    \end{figure}
\end{itemize}

\subsection{Fermions and Bosons}

\begin{itemize}
    \item To begin, we distinguish \textbf{fermions} and \textbf{bosons} by their obedience to the Pauli Exclusion Principle: fermions follow the PEP, while bosons do not.
    \item Now, we can establish that two particles are identical if they have the same intrinsic properties, such as mass, charge, and spin.
    \item In quantum mechanics (what we are concerned about), two identical particles are indistinguishable. We can now state that for two spinless particles:
    \begin{center}
        $\big|\psi(x_1,x_2)\big|^2dx_1dx_2$ = $\begin{cases}
            \text{probability of finding two particles} \\ \text{between } x_{1,2} \text{ and } x_{1,2} + dx_{1,2} \text{, respectively}.
        \end{cases}$
    \end{center}
    \item However, if they have spin:
    \begin{center}
        $\big|\psi(x_1, m_1, x_2, m_2)\big|^2dx_1dx_2$ = $\begin{cases}
            \text{probability of finding a particle between } \\ x_1 \text{ and }x_1 + dx_1 \text{with } S_z = m_1\hbar \text { and another} \\
            \text{between } x_2 \text{ and } x_2 + dx_2 \text{ with } S_z = m_2\hbar.
        \end{cases}$
    \end{center}
    \item To determine the property of the particle (fermion or boson), we use this property:
    \begin{itemize}
        \item[$\blacksquare$] $\psi(2,1) = +\psi(1,2) \longrightarrow$ wavefunctions satisfying this property are \textbf{bosons}, which are symmetric, such as a photon (s = 1) or pion (s = 0). 
        \item[$\blacksquare$] $\psi(2,1) = -\psi(1,2) \longrightarrow$ wavefunctions satisfying this antisymmetric property are \textbf{fermions}, such as an electron, proton, or neutron. 
    \end{itemize}
    \item Two identical bosons can occupy the same one-particle state.
    \item Let's consider a two-particle wave functions with two non-identcal particles, such as one proton, one electron.
    \begin{equation*}
        \psi(1,2) = \phi(1)\chi(2)
    \end{equation*}
    where $\phi$ is the state of the proton and $\chi$ is the state of the electron.
    \item Now consider two identical fermions with combined wavefunction:
        \begin{equation*}
            \psi(1,2) = \phi(1)\chi(2) - \phi(2)\chi(1)
        \end{equation*}
    \begin{itemize}
        \item[$\blacksquare$] This equation satisfies $\psi(2,1) = -\psi(1,2)$; also note that if $\phi = \chi$, then $\psi = 0$, meaning there are \textbf{no wavefunctions where} ${\phi}$ and ${\chi}$ can occupy the same state. This is the whole basis of the Pauli's Exclusion Principle.  
    \end{itemize}
    \item Now consider two identical bosons with combined wavefunction:
    \begin{equation*}
        \psi(1,2) = \phi(1)\chi(2) + \phi(2)\chi(1)
    \end{equation*}
    \begin{itemize}
        \item[$\blacksquare$] This equation satisfies $\psi(2,1) = +\psi(1,2)$; also note that if $\phi = \chi$, then $\psi \neq 0$, meaning two particles can occupy the same state, which violates Pauli's Exclusion Principle.  
    \end{itemize}
\end{itemize}

\subsection{Ground States of the First Few Elements}

\begin{itemize}
    \item The ground state of the hydrogen electron (1s state) is $E = -E_R = -13.6 eV$.
    \begin{itemize}
        \item[$\blacksquare$] The \textbf{ionization energy} is the energy required to remove electron \textbf{---} $13.6 eV$.
    \end{itemize}
    \item The energy of electron in the 1s state of helium is $E = -Z_{\text{eff}}^2E_R$.
    \begin{itemize}
        \item[$\blacksquare$] The first ionization energy of helium is $24.6eV$, where $Z_{\text{eff}} \approx \sqrt{2}$.
        \item[$\blacksquare$] $E_{\text{ion}} \propto Z_{\text{eff}}^2$.
    \end{itemize}
    \item The \textbf{first excitation energy} is the energy needed to lift an electron to its first excited state $(n = 1 \rightarrow 2)$.
    \begin{itemize}
        \item[$\blacksquare$] $E_{\text{excite}} \approx Z_{\text{eff}}^2 \rightarrow$ for helium $(Z = 2)$, $E_{\text{excite}} = 19.8 eV$, whereas for hydrogen $(Z = 1)$, $E_{\text{excite}} = 10.2 eV$.  
    \end{itemize}
    \item Higher ionization and excitation energy indicates a high atomic stability and low chemical activity. Helium has the highest ionization and excitation energy of any atom.
    \item In terms of size, the radius of helium is about $\frac{1}{Z_{\text{eff}}}$ times that of hydrogen. 
    \item For lithium $(Z=3)$, $Z_{\text{eff}} \approx 1$, which is the same as the electron in the 1s state of hydrogen. As such, its ionization energy is:
    \begin{equation*}
        E_{\text{ion}} \approx E_{2s, H} = \frac{E_R}{n^2} \approx 3.4eV
    \end{equation*}
    Its true value is actually around $5.4eV$ but that is because $Z_{\text{eff}}$ is not exactly 1.
    \item By increasing Z, the radius of the atom shrinks, thereby \textbf{increasing ionization energy}.
    \begin{figure}[!ht]
        \centering
        \includegraphics{ionizatin.png}
        \caption{This trend shows that $E_{\text{ion}} \propto \frac{1}{r}$.}
    \end{figure}
    \item \textbf{Electron affinity} is the tendency of an atom to bind an extra electron and can be measured by the energy released when an atom captures an electron.
\end{itemize}
\vfill\pagebreak

\subsection{The Remaining Elements}

\begin{figure}[!ht]
    \centering
    \includegraphics{ionization.png}
\end{figure}

\begin{itemize}
    \item The total spin ($\sum \textbf{S}$) and orbital momentum ($\sum\textbf{L}$) of a \textbf{filled valence shell} is always zero.
\end{itemize}

\section{Molecules}

\subsection{Molecular Properties}

\begin{itemize}
    \item Ordinary matter is electrically neutral: increasing electrostatic interaction increases strength in solids.
\end{itemize}
\textbf{Noble Gases:}
\begin{itemize}
    \item These are Column 18 elements, where the full valence shell is filled.
    \item They have \textbf{no stable molecules} because of its reluctance to mingle; the electron affinity is approximately 0. As such, the electron distribution cannot easily be changed because excited states are very high in energy.
    \item When a noble atom is close to another noble atom, there is no force between the two because they are spherically symmetric. As such, no external electric field will be produced and no net force will be experienced when placed in an applied field.
\end{itemize}
\textbf{Ionic Bonding:}

\begin{itemize}
    \item The strongest electrostatic attraction is between two oppositely charged stable spheres, this being lithium and fluorine.
    \item Since the ions are both spherically symmetric, they can be treated like point charges, forming an \textbf{electric dipole}. The \textbf{dipole moment} of equal but opposite charges is defined to be:
    \begin{equation*}
        p = qd
    \end{equation*}
\end{itemize}

\textbf{Covalent Bonding:}

\begin{itemize}
    \item Some diatomic molecules have dipole moments much smaller than ionic bonds. In this case, their valence electrons are shared.
    \begin{figure}[!ht]
        \centering
        \includegraphics{covalent.png}
        \caption{The shaded region is where the wavefunctions between the two atoms interfere constructively. There is a concentration of charge in the region between the two nuclei.}
        \label{fig:my_label}
    \end{figure}
\end{itemize}

\textbf{Estimating Bond Strength:}

\begin{itemize}
    \item For an ionic molecule, we can make the assumption that the electron is fully transferred from one molecule to the next, and the potential energy between the two ions at separation R is:
    \begin{equation*}
        U(r) = -\frac{ke^2}{R}
    \end{equation*}
    \item Since we know that potential energy is the majority of the molecule's total energy, we can estimate that $E \approx U(r)$, meaning the binding energy $B = -E \approx -U$.
    \item This is similar for a covalent molecule by making similar assumptions regarding the distribution of charge in the molecule.
\end{itemize}
\subsection{The Ionic Bond}

\begin{itemize}
    \item An \textbf{ionic bond} is formed when one electron is transferred from one atom to another, producing a pair of oppositely charged ions with strong attraction between them.
    \begin{itemize}
        \item[$\blacksquare$] This occurs when the first atom has a low $E_{\text{ion}}$ and the other atom has a high electron affinity.
    \end{itemize}
\end{itemize}

\textbf{Example: NaCl}

\begin{itemize}
    \item If we were to determine the energy needed to transfer an electron from the Na to the Cl if the atoms were far apart, it would cost energy. This is to say $E(R) = \Delta E$
    \item However as they come closer, potential energy comes into play. This is to say $E(R) \neq \Delta E$.
    \begin{align}
    \vcenter{\hbox{\includegraphics[width=7cm, height=3cm]{nacl.png}}}
        &\qquad\qquad
        \begin{aligned}
             E(R) = \Delta E = E_{ion}(Na) - EA(Cl) = 1.5eV \\
             E(R) = \Delta E - U(R) = \Delta E - \frac{ke^2}{R} \\
             (R > R_c): \hspace{0.5 cm} E_{[Na^+ + Cl^-]} > E_{[Na + Cl]}
        \end{aligned}\\
        \vcenter{\hbox{\begin{minipage}{7cm}
        \captionof{figure}{As $R \rightarrow \infty$, $E(R) = \Delta E$. However beyond $R_c$, the critical radius, the energy will be less than $E(R)$ due to the potential energy.}
        \end{minipage}}}
        & \notag
    \end{align}
    \item As seen above, once $R < R_c$, the ions have less energy than the neutral atoms since the potential energy, $U(R)$ offsets the costs $\Delta E$ of transferring the electron.
    \item Interestingly, the energy does not continue to decrease as $R \rightarrow 0$ since once the ions overlap, they begin to \textbf{repel each other}, causing the energy to rise once more.
    \begin{figure}[!ht]
        \centering
        \includegraphics{nacl2.png}
        \caption{At $R<R_0$, repulsive forces dominate and $E(R)$ rises quickly. $E(R_0) = -B$, the binding energy (minimum).}
        \label{fig:energy}
    \end{figure}
    \item We can safely predict the values of $R_0$ (center-to-center distance of the two atoms in the lowest state of the stable molecule) and B:
    \begin{equation*}
        B = -E(R_0) \approx \frac{ke^2}{R_0} - \Delta E
    \end{equation*}
\end{itemize}

\subsection{The Covalent Bond}

\begin{itemize}
    \item A \textbf{covalent bond} involves the concentration of one or more electrons from each atom in the region between the two nuclei; sharing of electrons.
\end{itemize}

\subsubsection{$\mathbf{H_2^+}$ Molecular Ion}

\begin{itemize}
    \item In $H_2^+$, the single present electron moves in the field of the two protons, but we can treat the protons as stationary since $m_{p^+} >> m_{e^-}$.
    \item Solving Schr\"{o}dinger's Equation, we can obtain its lowest allowed energy E, which is, as established previously, dependent on R. Similar to ionic bonds, if $E(R)$ has a minimum at some separation $R_0$, a stable molecule will exist with bond length $R_0$ and energy $E(R_0)$. 
    \item We can use the following diagram to denote the electron's position relative to the two protons, denoted $\vec{r}$. This is used to find the wavefunction of the electron that gives us minimum energy with separation R.
    \begin{figure}[!ht]
        \centering
        \includegraphics{pep.png}
    \end{figure}
    \begin{align}
    \vcenter{\hbox{\includegraphics[width=6cm, height=3.5cm]{p1p2.png}}}
        &\qquad\qquad
        \begin{aligned}
             \psi_1(\vec{r}) = Ae^{-r_1/a_B} \text{ (e- close to p1)}\\
             \psi_2(\vec{r}) = Ae^{-r_2/a_B} \text{ (e- close to p2)}
        \end{aligned}\\
        \vcenter{\hbox{\begin{minipage}{6cm}
        \captionof{figure}{We are assuming that the protons are far apart, or $R >> a_B$.}
        \end{minipage}}}
        & \notag
    \end{align}
    \item Now since we know that the equations above solve for the wavefunction yielding the same energy, any linear combination ($\psi$, generalized) will also be degenerate:
    \begin{equation*}
        \psi = B\psi_1 + C\psi_2
    \end{equation*}
    \item We have to consider 2 states (symmetric and antisymmetric) of the wavefunction since they are both valid solutions to Schr\"{o}dinger's Equation:
    \begin{align}
    \vcenter{\hbox{\includegraphics[width=10cm, height=4cm]{psi12.png}}}
        &\qquad\qquad
        \begin{aligned}
            \psi_+ = \psi_1 + \psi_2 \\
            \psi_- = \psi_1 - \psi_2 \\
            |\psi(x)|^2 = |\psi(-x)|^2
        \end{aligned}\\
        \vcenter{\hbox{\begin{minipage}{10cm}
        \captionof{figure}{The right-hand plot shows how the probability density of the electron, $|\psi_{\pm}|^2$, is identical regardless of the state.}
        \end{minipage}}}
        & \notag
    \end{align}
    \item The individual wavefunctions are known as \textbf{atomic orbitals}, while the combinations of $\psi_+$ and $\psi_-$ are called \textbf{molecular orbitals ---} describe states where the electron is associate with both nuclei.
    \item When the two protons approach each other, the two wavefunctions will overlap, causing the electron to be influenced by both protons simultaneously. However, $\psi_{\pm} = \psi_1 \pm \psi_2$ still holds under reasonable assumptions due to symmetry properties.
    \begin{figure}[!ht]
        \centering
        \begin{subfigure}{0.48\textwidth}
            \centering
            \includegraphics{p1p2 3.png}
            \caption{Symmetric state results in \textbf{constructive interference} and increased electron density between the nuclei. Antisymmetric state results in \textbf{destructive interference} and decreased electron density.}
            \label{fig:p1p2}
        \end{subfigure}%
        \hspace{0.2cm}
        \begin{subfigure}{0.48\textwidth}
            \centering
            \includegraphics[width=0.7\textwidth]{p1p2 2.png}
            \caption{This figure explains the effect of Figure \ref{fig:p1p2}, showing that the density of electrons at the origin (maximum overlap) is zero for an antisymmetric state (bottom), whereas it is enhanced in the symmetric state (top).}
            \label{fig:p1p2 2}
        \end{subfigure}    
    \end{figure}
    \item Using all of the information from Figures \ref{fig:p1p2} and \ref{fig:p1p2 2}, we can determine the trend of the energy given a symmetric or antisymmetric state. 
    \begin{itemize}
        \item[$\blacksquare$] For the symmetric state, known now as the \textbf{bonding orbital} $(\psi_+)$, the energy decreases as the protons get closer until it reaches a minimum at $R = R_0$. Once $R < R_0$, the dominant force is the repulsive force of the two protons.
        \item[$\blacksquare$] For the antisymmetric state, known now as the \textbf{antibonding orbital} $(\psi_-)$, the energy increases steadily as R decreases and has no minimum, meaning that there is no stable bound state for this state.
        \begin{figure}[!ht]
            \centering
            \includegraphics{energy.png}
            \caption{The curve $E_+$ represents the bonding orbital while the curve $E_-$ represents the antibonding orbital.}
            \label{fig:my_label}
        \end{figure}
    \end{itemize}
\end{itemize}

\subsubsection{Covalent Bonding of Multielectron Atoms}
\begin{itemize}
    \item The formation of molecules is pretty simple; only the valence electrons will overlap enough to have an important role in the molecular bond. Now all we have to recall is that the $2p$ orbital has 3 independent wavefunctions.
    \begin{itemize}
        \item[$\blacksquare$] \textbf{Example (Fluorine):} Fluorine has 5 valence electrons with electron configuration:
        \begin{equation*}
            F: 1s^22s^22p^5
        \end{equation*}
        \item[$\cdot$] This can be rewritten, taking into account the orientations of the orbitals:
        \begin{equation*}
            F: 1s^22s^22p_x^22p_y^22p_z^1
        \end{equation*}
        \item[$\cdot$] It does not matter which orbital is filled ($2p_x$, $2p_y$, or $2p_z$), the point is the two that are filled will have antiparallel spins and cannot participate in a covalent bond. This leaves only 1 electron available to form a covalent bond.
    \end{itemize}
\end{itemize}
\subsection{Constructing Molecular Bonds}

\begin{itemize}
    \item For the molecular orbital energy level diagram, it is directly related to the bonding and antibonding orbitals.
    \item For the 1s level (\textbf{n = 1}), the antibonding orbitals ($\sigma(1s)^*$) are placed on top, while the bonding orbitals ($\sigma(1s)$) are placed on the bottom.
    \begin{align}
    \vcenter{\hbox{\includegraphics[width=10cm, height=4cm]{MO.png}}}
        &\qquad\qquad
        \begin{aligned}
            BO = \frac{\# \text{ of BO} - \# \text{ of AO}}{2}
        \end{aligned}\\
        \vcenter{\hbox{\begin{minipage}{10cm}
        \end{minipage}}}
        & \notag
    \end{align}
    \item For (\textbf{n = 2}), the LCAO-MO (linear combination of atomic orbital-molecular orbital) overlap to forms to MO's. $\sigma$ bonds are formed by the overlap between $2s$ and $2p_z$ atomic orbitals, while $\pi$ bonds are formed by the overlap of two $2p$ bonds.
    \begin{figure}[!ht]
        \centering
        \includegraphics{bond.png}
        \caption{Only same orientation $2p$ orbitals will affect the electron density at the overlap. Orthogonal p orbitals cannot mix to increase electron density, such as $p_x + p_y$.}
        \label{fig:my_label}
    \end{figure}
    \item For \textbf{homonuclear atoms} (such as $N_2$), contribution from both atoms are same. For \textbf{heteronuclear atoms} (such as $NO$), more contribution from one atomic orbital than another: more ionic and less covalent. \textbf{More electronegative means less energy}.
    \begin{figure}[!ht]
        \centering
        \includegraphics{elec.png}
        \caption{The lower energy atom in bonding pair will contribute the most amplitude in electron probability distribution.}
    \end{figure}
\end{itemize}

\subsubsection{Example: NO}

\begin{itemize}
    \item For nitric oxide, the ground state configuration can be simplified to:
    \begin{equation*}
        NO: \sigma(2s)^2\sigma^*(2s)2\pi(2p_x,p_y)^4\sigma(p_z)^22\pi(2p_x,p_y)^{*1}
    \end{equation*}
    \begin{figure}[!ht]
        \centering
        \includegraphics[scale = 0.5]{no.png}
    \end{figure}
\end{itemize}
\subsection{VESPR}

\begin{itemize}
    \item This is a theory used to predict the geometry of individual molecules from the number of electron pairs surrounding the central atoms.
    \begin{figure}[!ht]
        \centering
        \includegraphics{vespr.png}
    \end{figure}
\end{itemize}

\subsubsection{$\mathbf{sp^3}$ Hybridization}

\begin{itemize}
    \item Formation of $sp^3$ hybrid orbitals by combination of an s orbital and three p orbitals, creating a formation of four lobes with an angle of \textbf{109.5}$^{\circ}$.
    \item Either 4 bonds ($CH_4$), 3 bonds + 1 lone pair ($NH_3$) or 2 bonds + 2 lone pairs ($H_2O$).
    \begin{figure}[!ht]
        \centering
        \includegraphics{sp3.png}
    \end{figure}
    \item Filling out electron configuration in $sp^3$ hybrid is different than normal:
    \begin{figure}[!ht]
        \centering
        \includegraphics{sp3 2.png}
        \caption{$CH_4$}
    \end{figure}
\end{itemize}

\subsubsection{$\mathbf{sp^2}$ Hybridization}
\begin{itemize}
    \item Combination of one s orbital and 2 p orbitals, results in 3 $sp^2$ orbitals with bond angle of $\mathbf{120^{\circ}}$.
    \item $sp^2$ bonds can form both $\sigma$ and $\pi$ bonds depending on the linear combination with the unhybridized p orbital.
    \begin{figure}[!ht]
        \centering
        \begin{subfigure}{0.48\textwidth}
            \centering
            \includegraphics[width=0.7\textwidth]{sigma and pi.png}
            \caption{Sideways overlap forms a $\pi$ bond, head-on overlap forms  $\sigma$ bond.}
            \label{fig:p1p2}
        \end{subfigure}%
        \hspace{0.2cm}
        \begin{subfigure}{0.48\textwidth}
            \centering
            \includegraphics[width=0.9\textwidth]{ethylene.png}
            \caption{Example of ethylene ($H_2C = CH_2$) with a combination of $\sigma$ and $\pi$ bonds.}
            \label{fig:p1p2 2}
        \end{subfigure}    
    \end{figure}
\end{itemize}
\subsubsection{sp Hybridization}

\begin{itemize}
    \item Combination of one s and one p orbital gives two sp hybrid orbitals with bond angle of $\mathbf{180^{\circ}}$. The two unhybridized orbitals remain and are oriented at $90^{\circ}$ angles to the hybrids.
    \item This hybridization results in minimum electron repulsion.
    \begin{figure}[!ht]
        \centering
        \includegraphics{sp orbital.png}
        \caption{Can make 2 $\pi$ bonds from the unhybridized orbitals.}
    \end{figure}
\end{itemize}

\section{Intermolecular Forces}

\begin{itemize}
    \item \footnote{RJDM Chapter 4.1, pg.93}Quantum mechanics effects everything. Even rotational motions of molecules show quantization effects due to underlying wave properties and continuous force relationships. However, rotational motion travels at the lowest frequency and so the energy levels become so closely spaced that we can treat them classically, known as the \textbf{Correspondence Principle}.
    \item Any variable affecting the energy of the system can be extrapolated to the classical limit, $\Delta E << kT$.
    \begin{figure}[!ht]
        \centering
        \includegraphics{inter vs intra.png}
        \caption{Intra. (bonding) versus Inter. (non-bonding) forces. Bonding forces are typically long range and require more energy, while nonbonding are short range less energy.}
        \label{fig:intervsintra}
    \end{figure}
    \item For intermolecular forces, the potential energy curves all look very similar, all similar to Figure \ref{fig:energy}. Nothing breaks in compression, not even molecules. Shear force is the main contributor. As we already know, force is \textbf{always} the gradient of the operating potential.
    \begin{equation*}
        F = -\nabla U(r) 
    \end{equation*}
    \begin{figure}[!ht]
        \centering
        \includegraphics{foce.png}
        \caption{The repulsive potential is due to n-n and e-e repulsion overcoming the attractive potential, scaling with a factor of $\frac{1}{r^n}, n \in [9,12]$.}
        \label{fig:my_label}
    \end{figure}
\end{itemize}

\subsection{Metallic Bonding}

\begin{itemize}
    \item \textbf{Metallic bonds} describe covalent bonding with different degrees of ionic character, typically involving $n > 3$ periods.
\end{itemize}
\subsection{Ion-Dipole Interaction}

\begin{itemize}
    \item This force involves a fully unscreened ion (anion or cation with
    specific charges $\pm1, \pm2, \pm3$ etc. possible depending on ion) coupled through Coulombic interaction to a polar molecule with a permanent dipole moment.
    \item This interaction requires a large difference in electronegativities between bonded atoms.
    \item The potential of this intermolecular force is on the order of $U(r) \propto \frac{1}{r^6}$.
\end{itemize}

\subsection{Hydrogen Bonding}

\begin{itemize}
    \item This force arises only when hydrogen is bonded with oxygen, fluorine, or nitrogen. Its specificity is the reason for the directional natures of biological molecules.
    \item This potential also scales on the order of $U(r) \propto \frac{1}{r^6}$.
\end{itemize}

\subsection{Dipole-Dipole Interaction}

\begin{itemize}
    \item This force occurs between two polar molecules, where the molecules will arrange itself such that the positive end of a polar molecule will be attracted to the negative end of a molecule. This will lead to solid state structures, often repeating to create the crystal lattice structure.
\end{itemize}
\subsection{Ion-Induced Dipole Interaction}

\begin{itemize}
    \item This force occurs when an ion approaches a molecule/atom with uniform charge distribution, causing the electrons within the molecule/atom to be either attracted to or repelled by depending on the charge of the ion. This induces a dipole in the atom.
    \item The potential energy of this force scales on the order of $U(r) = \frac{1}{r^4}$.
\end{itemize}

\subsection{Dipole-Induced Dipole Interaction}
\begin{itemize}
    \item Same idea as ion-induced dipole, but with a dipole in place of an ion. It will polarize the adjacent atom, depending on its orientation.
\end{itemize}

\subsection{Van der Waals Force}
\begin{itemize}
    \item The weakest, shortest force and it occurs between molecules with uniform charge distributions.
    \item At a standstill snapshot, the electrons will arrange themselves such that they are not evenly distributed throughout the molecule despite it being electrically neutral. So when two molecules get close to each other, dipoles will temporarily be induced between the molecules.
\end{itemize}

\subsection{The Hydrogen Bond}

\begin{itemize}
    \item Hydrogen bonds arise due to the induction of electron density and the creation of a $\delta^+$ charge on the hydrogen atom. The hydrogen atom will lose a "tug-of-war" battle for electrons to more electronegative atoms (N, O, F).
    \item Hydrogen bonding explains the high boiling/melting point in water compared to other liquid substances. Hydrogen bonds require more energy to overcome in comparison other intermolecular forces.
\end{itemize}

\section{Spectroscopy}

\subsection{Photon and the Quantization of Light}

\begin{itemize}
    \item The energy of the photon ($E = h\nu$) absorbed or emitted must match the change in internal energy of the molecules or material involved. Since we have exact solutions to the hydrogen atom, we can use it to draw conclusions on the interaction of light with matter.
    \item Both absorption and emission indicate the energy difference between two quantized energy levels:
    \begin{equation*}
        \Delta E = E_m - E_n = h\nu
    \end{equation*}
    \begin{figure}[!ht]
        \centering
        \includegraphics[scale = 0.9]{absorption.png}
    \end{figure}
    \item Light is the propagation of an electromagnetic field. It is the electric field component to light that can interact with the electrons and change their distribution. 
    \item It is possible to promote an electron in an atom or molecule to a higher energy level by sending light to it, if the energy from the light is sufficient.
    \begin{figure}[!ht]
        \centering
        \includegraphics[scale = 0.65]{light.png}
        \caption{Sending an electron from a bonding orbital to an antibonding orbital will weaken the chemical bond.}
    \end{figure}
    \item Now recall the Hamiltonian operator. We have to make modifications to potential $U(r)$ because there is potential when an electric field interacts with a dipole:
    \begin{equation*}
        U(r) = -\frac{ke^2}{r} + \mathbf{\mu E(t)}
    \end{equation*}
    \item The probability of transitioning energy states, defined by $\mu_{mn}$ meaning $\mathbb{P}(n \rightarrow m)$ is:
    \begin{equation}
        \mu_{mn} = \int_{-\infty}^{\infty}\psi_m\vec{\mu}\psi_nd\tau
    \end{equation}
    where $\tau$ is defined as all of space.
    \item All transitions must involve a change in symmetry, leading to the understanding that $\mathbf{\Delta}l = \pm 1$. This means an atom cannot transition from a 1s state to its 2s state, must transition from 1s to 2p otherwise it will not induce a dipole moment.
\end{itemize}
\subsubsection{Selection Rules for Transitions}
\begin{enumerate}
    \item \textbf{Spin selection rule:} changes in spin multiplicity are forbidden.
    \begin{equation*}
        \Delta S = 0
    \end{equation*}
    This means the allowed transitions are:
    \begin{itemize}
        \item[$\blacksquare$] singlet (S = 0) $\rightarrow$ singlet
        \item[$\blacksquare$] triplet (S = 1) $\rightarrow$ triplet
    \end{itemize}
    \item \textbf{Laporte selection rule:} there must be a change in symmetry.
    \item \textbf{Angular momentum selection rule:} $\Delta l = \pm1$
\end{enumerate}
\subsubsection{Absorption and Emission (Beer-Lambert Law)}

\begin{itemize}
    \item Absorbance can be quantified if we know the incident light intensity, $I_0$ using the Beer-Lambert law:
    \begin{equation}
        A = -\log T =  -\log(\frac{I}{I_0}) = \varepsilon bC
    \end{equation}
    where b is the thickness of the medium, T is the transmittance.
    \item As such, it can be noted that absorbance and transmittance have an inversely, logarithmic relationship:
    \begin{equation}
        A = -\log T \rightarrow T = 10^{-A}
    \end{equation}
\end{itemize}

\subsection{Quantum Mechanic Harmonic Oscillators}
\begin{itemize}
    \item A vibrating molecule behaves like 2 masses joined by a spring \textbf{---} a spring system. As such, we can use Hooke's law to find the potential of the system.
    \begin{align*}
        F &= -kx \\
        U &= -\frac{dF}{dx} = \frac{1}{2}kx^2
    \end{align*}
    \item For two masses joined together by a spring, their effective mass $\mu$ vibrates with a frequency $\nu$, defined as:
    \begin{equation*}
        \mu = \frac{m_am_b}{m_a + m_b}; \nu = \frac{1}{2\pi}\sqrt{\frac{k}{\mu}}
    \end{equation*}
    \item The potential energy of a molecule behaving like a harmonic oscillator is
    \begin{align}
    \vcenter{\hbox{\includegraphics[width=7cm, height=6cm]{QMSH.png}}}
        &\qquad\qquad
        \begin{aligned}
             V = \frac{1}{2}k(r-r_e)^2 \\
             E_{vib} = (n + \frac{1}{2})h\nu,\hspace{0.25cm} n = 0, 1, ...
        \end{aligned}\\
        \vcenter{\hbox{\begin{minipage}{7cm}
        \end{minipage}}}
        & \notag
    \end{align}
    \item In comparison to real molecular potential, it is missing a factor due to anharmonic character:
    \begin{equation}
        E_{vib} = (n +\frac{1}{2})h\nu - x(n +\frac{1}{2})^2h\nu
    \end{equation}
    with a selection rule of $n = \pm1$
\end{itemize}
\subsubsection{Infrared Active Molecules (IR Active)}
\begin{itemize}
    \item The condition to be \textbf{IR Active} is if the electric dipole changes with bond length during a vibration.
    \begin{equation*}
        \frac{d\mu}{dr} > 0
    \end{equation*}
    \item This means if a molecule is symmetric, such as a diatomic molecule ($N_2$), it can only vibrate in one mode, meaning it is not IR Active.
\end{itemize}
\subsection{Franck-Condon Principle}
\begin{itemize}
    \item The Franck-Condon principle states that in a molecule, an electronic transition occurs much faster than nuclear motion. This means that during an electronic transition, the nuclei do not move significantly from their original positions. Therefore, the probability of a molecule undergoing an electronic transition depends on the overlap between the vibrational wavefunctions of the initial and final electronic states.
    \item When a molecule absorbs or emits a photon, it will undergo a transition. It is most likely to occur when the wavefunctions of the initial and final state maximally overlap.
    \begin{figure}[!ht]
        \centering
        \includegraphics[scale=0.4]{vib trans.png}
        \caption{Vertical transition.}
    \end{figure}
\end{itemize}
\subsubsection{Possible Transitions for Excited Molecules}
\begin{itemize}
    \item[$\blacksquare$] \textbf{Radiative Transition:} photon is absorbed or emitted (solid line).
    \item[$\blacksquare$] \textbf{Non-Radiative Transition:} energy is transferred between different degrees of freedom of a molecule or to the surroundings.
    \item[$\blacksquare$] \textbf{Intersystem Crossing:} transition between singlet and triplet state, which is otherwise forbidden.
    \begin{figure}[!ht]
        \centering
        \includegraphics{transtiotn.png}
    \end{figure}
\end{itemize}

\noalign\hline
\section{Thermal Physics}
\begin{itemize}
    \item System implies thermodynamic system, which just means there is a lot of particles. To get an idea about general properties, we use the ideal gas, which has order of $\approx 10^{23}$ molecules.
    \item Thermodynamic equilibrium is achieved when pressure, temperature, and density are all \textbf{uniform} and \textbf{constant}.
    \item \textbf{Ideal Gas Law:}
    \begin{equation}
        pV = NkT
    \end{equation}
    \begin{itemize}
        \item[$\blacksquare$] at a fixed V, if T increases, so does P and vice versa.
        \item[$\blacksquare$] at a fixed T, if V increases, P decreases and vice versa.
    \end{itemize}
    \item We also know that $pV = Nm\bar{v}^2$, when we consider particles bouncing inside a box, where $\bar{v}^2$ is the square of the average velocity. We also concluded that the square of the average is equivalent to the average of the squared directional velocities ($\bar{v}^2 = \bar{v_{x,y,z}^2}$), meaning:
    \begin{equation*}
        kT = m\bar{v_x^2} = \frac{1}{3}m\bar{v^2} = \frac{2}{3}\frac{m\bar{v^2}}{2}
    \end{equation*}
    \item Thus in thermodynamic equilibrium, the average kinetic energy per particle in an ideal gas is $\frac{3}{2}kT$, known as the \textbf{equipartition theorem}. This energy can be broken down into three contributors, or degrees of freedom:
    \begin{enumerate}
        \item a translational DOF: $\frac{kT}{2}$
        \item a rotational DOF: $\frac{kT}{2}$
        \item a vibrational DOF: $kT$
    \end{enumerate}
    \item For a \textbf{linear}, n-atomic molecule, it has 3 translational, 2 rotational, and $3n - 5$ vibrational ($\sum DOF = 3n$) degrees of freedom.
    \item For a \textbf{non-linear}, n-atomic molecule, it has 3 translational, 2 rotational, and $3n - 6$ vibrational ($\sum DOF = 3n$) degrees of freedom.
    \item N n-atomic molecules (at T) has average energy U ($n \geq 3$):
    \begin{equation}
        U = N (\frac{3}{2}kT + \frac{3}{2}kT + (3n - 5/6)kT)
    \end{equation}
    depending if the molecule is linear or non-linear.
\end{itemize}
\subsubsection{Moving a piston}

\begin{itemize}
    \item This example is to demonstrate the \textbf{First Law of Thermodynamics}, and doing work:
    \begin{equation}
        \Delta U = Q - W
    \end{equation}
    where $\Delta U$ is the change in energy of the system and Q is the heat transferred to or from the system. If W is positive, it is work done by the system; if it is negative, it is work done on the system. It is also important to note that given a P-V diagram, the work done is its integral:
    \begin{equation}
        W = \int_{V_i}^{V_f}P dV
    \end{equation}
    \item If pressure is dependent on volume, we have a slightly different relationship:
    \begin{equation*}
        W = -\int_{V_i}^{V_f}P(V)dV = -NkT\int_{V_i}^{V_f}\frac{dV}{V} = -NkT\log{\frac{V_f}{V_i}}
    \end{equation*}
    This is the work done on an ideal gas during \textbf{isothermal expansion/compression}.
    \item If the gas is ideal, its energy will not change during an isothermal compression.
    \item We also know at every instance, the energy is defined as:
    \begin{equation*}
        U = \frac{f}{2}NkT
    \end{equation*}
    where $f = 3$ for a monatomic molecule, $f = 7$ for a diatomic molecule.
    \item \textbf{Heat capacity} is the measure of how much heat is needed to raise the T of an object by 1 K, defined colloquially as $C = \frac{Q}{\Delta T}$. However if V is fixed, we have:
    \begin{equation}
        c_V \equiv (\frac{Q}{\Delta T})_V = (\frac{\partial U}{\partial T})_V
    \end{equation}
\end{itemize}
\subsection{Main Postulate of Statistical Mechanics}

\begin{tcolorbox}[enhanced, boxrule = 0pt, frame hidden]
    In a closed (isolated: fixed-energy) system, all accessible microstates are \textbf{equally likely} in thermodynamic equilibrium.
\end{tcolorbox}
\begin{itemize}
    \item A \textbf{microstate} is one arrangement of the molecular position and kinetic energy of the system. Collisions change the microstate continuously, allowing the system to be in different microstates at different times.
\end{itemize}

\begin{tcolorbox}[enhanced, boxrule = 0pt, frame hidden]
    \textbf{For isolated systems:} all microstates are equally likely, meaning:
    \begin{equation}
        \mathbb{P}(\text{microstate}) = \frac{1}{\Omega(N, U)}
    \end{equation}
\end{tcolorbox}
\subsubsection{Physical Picture: The Paramagnet}

\begin{itemize}
    \item Picture in a crystal, every atom has a magnetic moment, $\vec{\mu}$, and this magnetic moment has a spin either up or down (like in quantum mechanics), $\vec{S} = \pm1$.
    \item One atom's internal energy will be defined as $u = -\vec{\mu}\cdot\vec{B}$, which is minimized when the RHS is maximized, or when $\mu$ is parallel to B.
    \item The total energy of the system, $U = -\mu_0B\sum_{i=0}^NS_i$ with N spins. One combination of the total energy is defined as the microstate of this system.
    \item If we define the total spin by the number of up-spins, $N_{up}$, as the macrostates, there are N macrostates:
    \begin{equation*}
        S = 2N_{up} - N, N_{up} = 0, 1, ..., N
    \end{equation*}
\end{itemize} 
\begin{tcolorbox}[enhanced, boxrule = 0pt, frame hidden]
    The total number of microstates with $N_{up}$ is defined as the number of ways to choose $N_{up}$ from N, namely the \textbf{multiplicity function}, or the number of microstates accessible to macrostate $N_{up}(S)$:
    \begin{equation}
        \Omega(N_{up}, N) = {N \choose N_{up}} = \frac{N!}{N_{up}!(N-N_{up})!}
    \end{equation}
\end{tcolorbox}
\subsubsection{Simple Harmonic Oscillator: The Einstein Solid}

\begin{itemize}
    \item An \textbf{Einstein solid} is a collection of N quantum SHO's. The microstate for this example is defined as $n_i$, which is the energy of one quantum SHO. The macrostate is defined as q, or the total energy of the system: 
    \begin{equation*}
        q = \sum_{i=0}^Nn_i
    \end{equation*}
    \item As defined in QM, the vibrational energy of a SHO is $E_{vib} \approx \hbar\omega n$, meaning the total energy of the system is $E_{tot} = \hbar\omega q$.
\end{itemize}
\begin{tcolorbox}[enhanced, boxrule = 0pt, frame hidden]
    The multiplicity function of an Einstein solid can be pictured as finding the total combinations of placing q balls in N boxes.
    \begin{equation}
        \Omega(N, q) = {N-1+q \choose q} = \frac{(N-1+q)!}{(N-1)!q!}
    \end{equation}
\end{tcolorbox}
\begin{itemize}
    \item This is helpful when considering combining two isolated systems together, with numbers ($N >> 1$) and energies, $q_A$ and $q_B$, with $q = q_A + q_B$.
    \item Some equations for this:
    \begin{align*}
        \mathbb{P}(\text{any microstate of A+B}) &= \frac{1}{\text{total no. of microstates of A+B with q}}\\
        \mathbb{P}(q_A') &= \frac{\Omega_A(q_A')\Omega_B(q-q_A')}{\text{total no. of microstates of A+B with q}}
    \end{align*}
\end{itemize}
\begin{tcolorbox}[enhanced, boxrule = 0pt, frame hidden]
    \textbf{Microscopic definition of entropy:}
    \begin{align}
        S(E, N, V) &= k\ln{\Omega(E,N,V)} \\
        \frac{1}{T} &= (\frac{\partial S(E, N, V)}{\partial E})_{N,V}
    \end{align}
\end{tcolorbox}
\begin{itemize}
    \item To find entropy of the Einstein solid, we must use Stirling's approximation for factorials:
    \begin{equation*}
        \ln{n!}|_{n\rightarrow\infty} \approx \ln{((\frac{n}{e})^n\sqrt{2\pi n})} = n(\ln{n} - 1) + \frac{1}{2}\ln{(2\pi n)}
    \end{equation*} 
    \item We can safely make the assumption that $N >> 1, q>>N$, which gives $\frac{q}{N} >> 1$. This assumption essentially means that on average there is large number of quanta per oscillator. After some calculation, we obtain that the multiplicity function simplifies to:
    \begin{equation*}
        \Omega(N, q)|_{N >> 1, q >>N} \approx (\frac{qe}{N})^N
    \end{equation*}
    \item Knowing that $S = k\ln{\Omega}$ and $E = \hbar\omega q$, we can manipulate these 3 equations to obtain the relationship:
    \begin{equation*}
        kT \approx \frac{\hbar\omega q}{N} \longrightarrow kT >> \hbar\omega
    \end{equation*}
    \item This is known as the high-T limit of the Einstein solid and equipartition only holds in this limit. This can be physically described as the internal energy of the system (kT) being much larger than the energy of a singular SHO ($\hbar\omega$).
\end{itemize}
\subsubsection{Returning to The Paramagnet}

\begin{itemize}
    \item Recall $U = -\mu_0BS = -\mu_0B(2N_{up} - N)$. How do we relate the total up-spin of the system with the entropy of the system? We note that $N_{up} = 0$ is $U_{max}, N_{up} = N$ is $U_{min}$, so we determine that for maximum multiplicity, $N_{up} = \frac{N}{2}$ (total spin = 0).
    \begin{figure}[!ht]
        \centering
        \begin{subfigure}{0.48\textwidth}
            \centering
            \includegraphics[width=0.7\textwidth]{U curve.png}
            \caption{Odd that U increases when multiplicity decreases.}
            \label{fig:p1p2}
        \end{subfigure}%
        \hspace{0.2cm}
        \begin{subfigure}{0.48\textwidth}
            \centering
            \includegraphics[width=0.9\textwidth]{S curve.png}
            \caption{For $U > 0, \frac{\partial S}{\partial U} = \frac{1}{T < 0}$, meaning negative T, and vice versa.}
            \label{fig:p1p2 2}
        \end{subfigure}    
    \end{figure}
\end{itemize}

\subsubsection{The Ideal Gas}

\begin{itemize}
    \item Recall the classical kinetic energy of an ideal gas:
    \begin{equation*}
        U = \sum_{i=1}^N\frac{\vec{p_i}^2}{2m}
    \end{equation*}
    \item We want to find how many ways we can choose the 3N numbers such that $\sum_{i=1}^N\vec{p_i}^2 = 2mU$ since the energy of the gas is dependent on the momenta of the atoms. This is essentially finding the multiplicity of the ideal gas. Intuitively, the energy increases as the number of choices increases.
    \item The multiplicity can be "dumbed down" to the area of a 3N-dimension sphere of radius-squared = 2mU. With the example of N = 1 (circle, 2D), we obtain that the area is:
    \begin{equation*}
        A_{3N-1}(R) = \frac{2\pi^{3N/2}}{(\frac{3N}{2}-1)!}R^{3N-1}
    \end{equation*}
    As such we expect the multiplicity function to be proportional to the above with R = $\sqrt{2mU}$, which does not yield a dimensionless number so we are not satisfied. After some factors the multiplicity function becomes:
    \begin{equation}
        \Omega(N,U) = \frac{1}{N!}\frac{1}{2^{3N}}\frac{2\pi^{3N/2}}{(\frac{3N}{2}-1)!}(\frac{\sqrt{2mU}L}{\pi\hbar})^{3N-1}
    \end{equation}
    \item This means that the momenta are quantized in units $\frac{\pi\hbar}{L}$..
    \item The entropy of this system (single-atomic ideal gas) is defined as:
    \begin{equation*}
        S(N,U,V) = k\ln{\Omega(N,U,V)} = Nk[\ln{(\frac{V}{N}(\frac{4\pi mU}{3Nh^2})^{\frac{3}{2}})} + \frac{5}{2}]
    \end{equation*}
    \item Using $c_V = \frac{\partial U}{\partial T}$ and $U = \frac{3}{2}NkT$, we can get $c_V = \frac{3}{2}Nk$.
    \item The condition that the total entropy was \textbf{extremized} yielded the thermodynamic equilibrium condition that:
    \begin{equation*}
        \frac{1}{T_1} = (\frac{\partial S_1}{\partial U_1})_{N_1,V_1} = \frac{1}{T_2} = (\frac{\partial S_2}{\partial U_2})_{N_2,V_2}
    \end{equation*}
    \item Now, we want to determine the mechanical equilibrium from the postulate of statistical mechanics. For a system with $S_1$ and $S_2$, $S_{1+2}$ should be extremized with respect to both $U_2'$ and $V_2'$. The total entropy of the system can be represented as:
    \begin{equation*}
        S_{1+2} = S_1(U - U_2', N_1, V - V_2') + S_2(U_2', N_2, V_2')
    \end{equation*}
    \item Extremizing the entropy would mean that we want $\nabla S|_{N}= 0$.
\end{itemize}

\begin{tcolorbox}[enhanced, boxrule = 0pt, frame hidden]
    Extremizing entropy is defined as finding the gradient of S with respect to $U_2'$ and $V_2'$
    \begin{equation}
        \nabla S = \frac{\partial S_{1+2}}{\partial U_2'} = 0 \rightarrow T_1 = T_2
    \end{equation}
    \begin{equation}
        \nabla S = \frac{\partial S_{1+2}}{\partial V_2'} = 0 \rightarrow (\frac{\partial S_1}{\partial V_1'}) = (\frac{\partial S_2}{\partial V_2'})
    \end{equation}
    \textbf{These are the two conditions to achieve mechanical equilibrium.}
\end{tcolorbox}
\begin{itemize}
    \item We can now statistically define pressure as:
    \begin{align}
        \frac{p_1}{T_1} &= (\frac{\partial S_1}{\partial V_1})_{N_1,U_1}\\
        p &= T(\frac{\partial S}{\partial V})_{N,U}
    \end{align}
    \item Rearrange this equation and compute and note that you will return to the ideal gas law with $\frac{p}{T} = \frac{Nk}{V}$.
    \item This helps us come to the formal definition of entropy with:
    \begin{align*}
        (\frac{\Delta S}{\Delta U}) &= \frac{1}{T} \\
        \Delta S &= \frac{\Delta U}{T} \approx \frac{Q}{T}
    \end{align*}
    from the First Law of thermodynamics
    \item The \textbf{Second Law of thermodynamics} states that the change in entropy is always positive (entropy is always increasing: $\Delta S \geq 0$).
    \item Consider two systems with $T_1$ and $T_2$, respectively. If the first system gives away heat Q, the total change in entropy can be calculated as:
    \begin{equation*}
        \Delta S = \Delta (S_1 + S_2) = Q(-\frac{1}{T_1}+\frac{1}{T_2}) \geq 0
    \end{equation*}
    \item We also know that by definition $c_V = \frac{\partial U}{\partial T}$, meaning that when we rearrange it $\Delta U = c_V\Delta T$. This way we can express change in entropy in another way.
    \begin{equation*}
        \Delta S = c_V\frac{\Delta T}{T} = \int_{T_i}^{T_f}c_V\frac{dT}{T}
    \end{equation*}
\end{itemize}

\subsection{Boltzmann Statistics}
\begin{itemize}
    \item Everything we have studied so far is known as the microcanonical distribution which refers to an isolated system with variables that were fixed. Thanks to Boltzmann, we have the canonical distribution, or the \textbf{Boltzmann distribution}, referring to a system where the temperature T is fixed rather than the energy E. Imagine a system being attached to a large reservoir system with a thermostat with energy $U_R$. The total energy U is equal to $U = U_R + E$, which is fixed because the thermostat system is isolated (fixed N, V). We can now apply the fundamental postulate of statistical mechanics stating that:
    \begin{equation*}
        \mathbb{P}(\text{any accessible microstate of system+R}) = \frac{1}{\Omega_{\text{system+R}}(U)}
    \end{equation*}
    \item Since we know that $U_R = U - E$, we say $\Omega_{\text{system+R}}(U) = \sum_E\Omega(E)\Omega_R(U - E)$, where E is every allowed eigenvalue of the Hamiltonian of the system (any allowed energy state when solving Schr\"{o}dinger's Equation). 
    \item Let us denote the energy of some particular state as $E_S$. The probability of the system being in such a microstate is equivalent to the probability that the reservoir can be in a microstate with energy $U - E_S$,  having exactly $\Omega_R(U - E_S)$ combinations. Mathematically:
    \begin{align*}
        \mathbb{P}(\text{system w/ } E_S) &= \mathbb{P}(\text{system+R in any microstate}) \times \Omega_R(U - E_S) \\
        \mathbb{P}(E_S) &= \frac{\Omega_R(U - E_S)}{\sum_E\Omega(E)\Omega_R(U - E)}
    \end{align*}
    \item Now consider the probability of two allowed energies ($E_{S1}, E_{S2}$):
    \begin{equation*}
        \frac{\mathbb{P}(E_{S1})}{\mathbb{P}(E_{S2})} = \frac{\sigma_R(U-E_{S1})}{\sigma_R(U-E_{S2})} = \frac{\text{exp}(\frac{S_R(U-E_{S1})}{k})}{\text{exp}(\frac{S_R(U-E_{S2})}{k})} = e^{\frac{1}{k}(S_R(U-E_{S1}) - S_R(U-E_{S2}))}
    \end{equation*}
    \item Since the reservoir is so much larger than the system, we are able to expand the expressions for the entropy:
    \begin{equation*}
        S_R(U-E_S) \approx S_R(U) - E_S\frac{\partial S_R(U)}{\partial U_R}
    \end{equation*}
    \item Now we can further expand the ratio of probabilities to:
    \begin{equation*}
        \frac{\mathbb{P}(E_{S1})}{\mathbb{P}(E_{S2})} = e^{\frac{1}{k}(E_{S1} - E_{S2})\frac{\partial S_R(U)}{\partial U_R}} = e^{-\frac{E_{S1}-E_{S2}}{kT}}
    \end{equation*}
    \item $\frac{\mathbb{P}(E_{S1})}{e^{-\frac{E_{S1}}{kT}}}$ can not depend on the energy of the microstate and be constant which is the same for all $E_S$:
    \begin{equation*}
        \mathbb{P}(E_S) = \frac{e^{-\frac{E_S}{kT}}}{Z}
    \end{equation*}
    \item What is Z? Recall that the microstate of the system is determined by quantum numbers and energy:
    \begin{equation*}
        E = (\frac{\pi\hbar}{L})^2\frac{1}{2m}\sum_{i=1}^N\vec{n_i}^2
    \end{equation*}
    \item We also know that summing the probability of all energies will be 1.
    \begin{equation*}
        \sum_E\mathbb{P}(E)\Omega(E) = \frac{1}{Z}\sum_E\Omega(E)e^{\frac{-E}{kT}}
    \end{equation*}
    \item We can rearrange for Z:
    \begin{equation*}
        Z = \sum_E\Omega(E)e^{\frac{-E}{kT}} \text{ or } \sum_n e^{\frac{-E(n)}{kT}}
    \end{equation*}
    \item The probability at a given T that a system has energy E can use Gibbs' Free Energy, $F(E) = E - TS(E)$:
    \begin{equation*}
        \mathbb{P}(\text{at T that a system has E}) = \frac{1}{Z}\sum_E\Omega(E)e^{\frac{-E}{kT}} = \frac{1}{Z}e^{\frac{S(E)}{k}-\frac{E}{kT}} = \frac{1}{Z}e^{-\frac{E - TS(E)}{kT}} = \frac{1}{Z}e^{-\frac{F(E)}{kT}}
    \end{equation*}
    The most likely value of E is where F(E) is at its minimum.
\end{itemize}
\subsubsection{Free Energy}

\begin{itemize}
    \item What does $F(T,V,N)$ determine?
    \begin{equation*}
        \Delta F = \Delta E - \Delta TS - T\Delta S = \Delta E - \Delta TS - T(\frac{\partial S}{\partial E}\Delta E + \frac{\partial S}{\partial V}\Delta V + \frac{\partial S}{\partial N} \Delta N) = -S\Delta T - p\Delta V + \mu\Delta N
    \end{equation*}
    \item As such, we have that for a ($T, V, N$) system, F is the \textbf{thermodynamic potential} meaning its partial derivatives determine S, p, $\mu$ in thermodynamic equilibrium.
    \begin{equation*}
        S = -\frac{\partial F}{\partial T}; p = -\frac{partial F}{\partial V}; \mu = \frac{\partial F}{\partial N}
    \end{equation*}
    \item We also know that for a process with fixed T, $\Delta F$ = $\Delta E - T\Delta S = Q + W - T\Delta S$. Now if the process is quasistatic, $Q = T\Delta S$, and as such $\Delta F = W$.  It is generally more useful to know that \textbf{minimizing F determines thermodynamic equilibrium state}.
    \item Relationship between F and constant Z is important since for a T, V, N system, $Z(T,V,N)$ determines all:
    \begin{equation}
        F = -kT\ln{Z(T,V,N)}
    \end{equation}
    \item We use F in more real applications because not all systems will be isolated. Open systems use F because S (entropy) cannot always be maxed, but instead, we can minimize F to achieve order.
\end{itemize}
\subsubsection{Einstein Solid (Z now)}
\begin{itemize}
    \item Interestingly, Z is easier to calculate than $\Omega$. The energy of a microstate in an Einstein solid is $E(q_1 + ... q_N) = \hbar\omega(q_1 + ... q_N)$. Now the Z value of the Einstein solid can be defined as:
    \begin{equation*}
        Z = \sum_\text{s: microstates}e^{\frac{-E_s}{kT}} = \sum_{q_1 = 0}^{\infty}...\sum_{q_N = 0}^{\infty}e^{-\frac{\hbar\omega(q_1 + ... q_N)}{kT}} = (\sum_{q_1 = 0}^{\infty}e^{-\frac{\hbar\omega}{kT}q_1})...(\sum_{q_N = 0}^{\infty}e^{-\frac{\hbar\omega}{kT}q_N})
    \end{equation*}
    Each individual sum represents the Z of a single harmonic oscillator. This means that for an Einstein solid of N oscillators, we have:
    \begin{equation*}
        Z_N = (Z_1)^N, Z_1 = \sum_{q=0}^{\infty}e^{-\frac{\hbar\omega}{kT}q} = \frac{1}{1-e^{-\frac{\hbar\omega}{kT}}}
    \end{equation*}
    \item General guideline, if a system has $E = E_1 + E_2$, then $Z = Z_1Z_2$. Another way to express $Z_1$ is:
    \begin{equation*}
        Z_1 = 1 + e^{-\frac{\hbar\omega}{kT}} + ... e^{-N\frac{\hbar\omega}{kT}}  = e^{-\frac{E_0}{kT}} + ... e^{-\frac{E_N}{kT}}
    \end{equation*}
    \item This shows us that states with $E >> kT$ will not contribute to thermodynamic equilibrium, whereas $E << kT$ will. At T = 0, we have energy $E_0$ and all states with $E > E_0$ have suppressed contributions due to the $e^{-\text{constant}}$ factor, meaning
    \begin{equation*}
        Z \approx e^{-\frac{E_0}{kT}}, T \rightarrow 0
    \end{equation*}
    \item At absolute zero (T = 0K), energy $E_0 = 0$, meaning $Z \rightarrow 1$ as $T \rightarrow 0$. However as $T \rightarrow \infty$, Z is the total number of microstates. As such, we can interpolate that Z is an approximation of \textbf{how many systems contribute to the thermodynamic equilibrium} physics.
    \item Back to the Einstein solid, we now have $Z = (\frac{1}{1 - e^{-\frac{\hbar\omega}{kT}}})^N$ and $\mathbb{P}(E_S) = \frac{1}{Z}e^{-\frac{E_S}{kT}}$. The expectation value (average) of the energy:
    \begin{equation*}
        <E> = \sum_SE_S\mathbb{P}(E_S) = \frac{1}{Z}\frac{\partial}{\partial(-\frac{1}{kT})}\sum_Se^{-\frac{E_S}{kT}} = \frac{\partial}{\partial(-\frac{1}{kT})}\ln{Z} = kT^2\frac{\partial}{\partial T}\ln{Z}
    \end{equation*}
    i have no idea why Z = $\ln Z$, but let's now consider $\frac{1}{kT} \equiv \beta$, meaning $Z = \sum_Se^{-\beta E_S}$ and $<E> = -\frac{\partial}{\partial\beta}\ln{Z}$. With $Z = (1 - e^{-\frac{\hbar\omega}{kT}})^{-N}$, we can rearrange the average energy formula to be:
    \begin{equation}
        <E> = \frac{N\hbar\omega}{e^{\frac{\hbar\omega}{kT}}- 1}
    \end{equation}
    \item If $kT << \hbar\omega$, we have $\frac{<E>}{N} \approx \hbar\omega e^{-\frac{\hbar\omega}{kT}}$. If $kT >> \hbar\omega$, $\frac{<E>}{N} \approx kT$. This are the expressions for the average energy per oscillator with the high and low temperature limits.
    \item The standard deviation of the energy $\sigma_E$ is $\sigma_E = kT\sqrt\frac{c_V}{k} ~ \sqrt{N}$, meaning $\frac{\sigma_E}{<E>} ~ \frac{1}{\sqrt{N}}$. From this we see that as $N \rightarrow \infty$, the fluctuations of the energy around the mean are vanishingly small.
\end{itemize}

\end{document}
