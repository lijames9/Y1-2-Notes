\section{Lecture 1}

\begin{itemize}
    \item In the everyday world, probability models are used to describe stochastic and random events.
    \item \textbf{What is random?} $\longrightarrow$ it is when an outcome of an event is uncertain.
\end{itemize}

\begin{defn}\textbf{Expectation}
    \begin{itemize}
        \item Structured randomness is categorized by \textbf{long-term stability}. For instance, if we repeat an experiment many times, where the results are $x_1$, $x_2$, ..., $x_N$, then the average
        \begin{equation*}
            \frac{1}{N}(x_1 + x_2 + ... + x_N)
        \end{equation*}
        will stabilize and converge to a limit as $N \rightarrow \infty$.
        \item This limit is called the \textbf{expectation of X (E[X])}.
    \end{itemize}
\end{defn}

\begin{rem}
    Traditional and studied probability models do not capture instances where the limit of the average as $N \rightarrow \infty$ does not converge to a value.
\end{rem}

\begin{itemize}
    \item Now, if:
    \begin{equation*}
        x_i = \begin{cases}
            1 \hspace{1.5cm} \text{if event E occurs} \\
            0 \hspace{1.5cm} \text{else}
        \end{cases}
    \end{equation*}
    then $\frac{1}{N}(x_1 + x_2 + ... + x_N)$ is equal to calculating the proportions of event E occurring.
    \item In this case, structural randomness means the frequency of event E occurring stabilizes as $N \rightarrow \infty$, and the limit is the probability of E.
\end{itemize}


\begin{rem}
    However, the expression of the average, $\frac{1}{N}(x_1 + x_2 + ... + x_N)$, is not a rigorous definition of expectation since the idea of repeating experiments to obtain an average holds no mathematical or rigorous value \textbf{--}
not enough to fixed elements in a repeated experiment to be rigorous.
\end{rem}
